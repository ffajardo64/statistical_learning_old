

\documentclass[12pt]{article}
%\documentclass[12pt]{scrartcl}
\nonstopmode
\usepackage{amsbsy,amsfonts,amsmath,amssymb,mathbbol,textcomp,bm}
\usepackage[top=1.5cm,left=2.0cm,right=2.0cm,bottom=2cm]{geometry}
\usepackage[tikz]{bclogo}
\usepackage{enumerate,syntonly,fancybox,booktabs,multirow,lastpage,fancyhdr,comment,url}
\usepackage[utf8]{inputenc}
\usepackage[T1]{fontenc}
\usepackage[brazilian]{babel}

\usepackage{hyperref}
\hypersetup{
    colorlinks=true,
    linkcolor=blue,
    filecolor=magenta,      
    urlcolor=blue,
    pdftitle={Overleaf Example},
    pdfpagemode=FullScreen,
    }

\urlstyle{same}

\pagestyle{fancy}

\fancyhf{}\fancyhead{} \fancyfoot{}
\fancyhead{}
\fancyfoot[L,L]{\copyright{}2022 Typeset \LaTeX}
\fancyfoot[R,R]{by F. Fajardo.}
\renewcommand{\headrulewidth}{0pt}

\DeclareMathSymbol{\minus}{\mathord}{operators}{"2D}
\newcommand{\overbar}[1]{\mkern 1.5mu\overline{\mkern-1.0mu#1\mkern-1.5mu}\mkern 1.5mu}
\DeclareMathOperator{\sen}{sen}


\begin{document}

\begin{center}
	\hrule
	\vspace{.4cm}
	{\textbf { \large Desafio da Semana --- Análise de Séries Temporais I}}
	\hrule
    \vspace{.4cm}
    \flushright{\today}
\end{center}

\parbox{\textwidth}{
\begin{center}
\fbox{\parbox{5.5in}{\small\centering
{\fontencoding{T1}\sffamily
\textbf{Instru\c c\~oes:} \begin{enumerate}[1.]
\item Leia atentamente a quest\~ao e responda \textbf{rigorosamente} cada item. Respostas \textbf{sem
justificativas} n\~ao ser\~ao consideradas;
\item As solu\c c\~oes dos desafios devem ser tipografadas no formato disponível no 
\href{https://ffajardo64.github.io/statistical_learning/STA13828/}{repositório do Github}. 
Para usuários do \texttt{RMarkdown}, solicita-se o uso do formato da \textit{ASA: American Statistical 
Association}, disponível no pacote \href{https://github.com/rstudio/rticles}{rticles} do \texttt{R}.
\textbf{Qualquer formato fora desses dois padrões será desconsiderado};
\item  As soluções devem ser encaminhadas em PDF no prazo estabelecido para entrega. \underline{Qualquer
entrega fora do prazo será desconsiderada};
\item A solução correta terá um valor de \textbf{1 ponto} na média final da Prova 1;
\item Lembre que, \textbf{as entregas não são obrigatórias}. Os discentes que não participarem dos desafios
não receberão qualquer tipo de punição. Porém, Encaminhamentos de qualquer tentativa incompleta, inacabada, 
inconclusa, incorreta, imprecisa ou mesmo ambígua, receberão uma punição de $\minus2$ (menos dois) pontos 
na média final da prova considerada acima;
\item Para este desafio, \textbf{será considerada para avaliação, única e exclusivamente, a 
primeira entrega recebida};
\end{enumerate}
}}}
\end{center}}

\input{desafio1} % Comente a linha quando tipografar a solução

\newpage

\begin{center}
\shadowbox{
\begin{minipage}{13cm}
\centering \textbf{\large Solução do Desafio da Semana} \\
Séries Temporais I - Data: \today.
\end{minipage}
}
\end{center}

\vspace{0.2in}

{\textbf{Nome:}\ Escreva aqui seu nome}

\vspace{0.1in}

\textbf{Matr{\'\i}cula:}\ Escreva aqui seu número de matrícula

\vspace{0.3in}

Escreva aqui a solução do desafio!


\end{document}

